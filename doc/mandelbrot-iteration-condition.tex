\documentclass[11pt]{article}
\usepackage{geometry}
\usepackage{amsmath}
\usepackage{amssymb}
\usepackage{amsthm}
\usepackage{bbold}
\geometry{noheadfoot}
\setlength{\topskip}{0mm}

\date{2014-07-25}
\author{Geoffrey T. Falk}
\title{Mandelbrot Iteration Condition}

\begin{document}
\pagenumbering{gobble}

\maketitle

\newtheorem*{lemma}{Lemma}

\begin{lemma}
Let $f(z) = z^q + c$, where $z, c \in \mathbb C$, $q > 1$.
If $|z| > \max(2^{1/(q-1)}, |c|)$, then
$|f^n(z)| \rightarrow \infty$ as $n \rightarrow \infty$.
\end{lemma}

\begin{proof}
Define a function $$\varepsilon(x) = \min\left(x^{q-1} - 2, 1 - \frac{|c|}{x}\right)$$
$\varepsilon$ is an increasing function of $x$, and $\varepsilon(|z|) > 0$.

We show that there is a constant $k$ (independent of $n$) such that 
$|f^n(z)| > |z| + n k$ for all $n \in \mathbb N$. In the case $n = 1$,

\begin{equation} \label{eq:e1}
\begin{aligned}
|f(z)| &= \left| z^q + c \right| = |z| \left| z^{q-1} + \frac{c}{z} \right| \\
&\geq |z| \left( |z|^{q-1} - {\left|\frac{c}{z}\right|}\right) \text{ by triangle inequality} \\
&\geq |z| \left( \left(2 + \varepsilon(|z|) \right) - \left(1 - \varepsilon(|z|)\right)\right)
 \text{ by definition of $\varepsilon$} \\
&= |z| \left(1 + 2 \varepsilon(|z|) \right) \\
&= |z| + 2 |z| \varepsilon(|z|) \\
&> |z| + 2^{1 + {1/(q-1)}} \varepsilon(|z|)
\end{aligned}
\end{equation}

Let $k = 2^{1 + {1/(q-1)}} \varepsilon(|z|) > 0$, so $|f(z)| > |z| + k$.

For the next step, assume the inductive hypothesis $|f^n(z)| > |z| + n k$.
Note $$|f^n(z)| > |z| > \max(2^{1/(q-1)}, |c|)$$.
Applying \eqref{eq:e1}, this time to $|f^n(z)|$,

\begin{equation} \label{eq:e2}
\begin{aligned}
\left|f^{n+1}(z)\right| 
&> \left|f^n(z)\right| +  2^{1 + {1/(q-1)}} \varepsilon(\left|f^n(z)\right|) \\
&> \left|f^n(z)\right| + 2^{1 + {1/(q-1)}} \varepsilon(|z|) \text{ since $\varepsilon$ is increasing} \\
&= \left|f^n(z)\right| + k \\
&> |z| + (n+1)k \text{ by the inductive hypothesis.}
\end{aligned}
\end{equation}
\end{proof}


\end{document}
